\documentclass[]{book}


\usepackage[utf8]{inputenc}
\usepackage[T1]{fontenc}
\usepackage[english]{babel}
\usepackage{kpfonts}
\usepackage[stretch=10,shrink=10]{microtype}

\usepackage[top=1.25in, bottom=1.25in,left=1.5in, right=1.25in]{geometry}
\setlength{\oddsidemargin}{36pt}
\setlength{\evensidemargin}{36pt}

%\usepackage{lmodern}
\usepackage[unicode=true]{hyperref}
\hypersetup{breaklinks=true,
            bookmarks=true,
            pdfauthor={},
            pdftitle={},
            colorlinks=true,
            citecolor=blue,
            urlcolor=blue,
            linkcolor=magenta,
            pdfborder={0 0 0}}
\urlstyle{same}  % don't use monospace font for urls
\setlength{\parindent}{0pt}
\setlength{\parskip}{6pt plus 2pt minus 1pt}
\setlength{\emergencystretch}{3em}  % prevent overfull lines
\setcounter{secnumdepth}{0}

\title{PRIO-GRID v.2.0 Codebook}
\author{Andreas Forø Tollefsen, Karim Bahgat, Jonas Nordkvelle and Halvard
Buhaug}
\date{\today}

\begin{document}

\frontmatter
\maketitle

\paragraph{Version Information:}\label{version-information}

This codebook describes the content and development of the PRIO-GRID
version 2.0.

\paragraph{Citation:}\label{citation}

PRIO-GRID is a unique data framework developed at PRIO and made freely
available to all interested users. Whenever using the PRIO-GRID data
frame or parts of its content, please cite:

\begin{quote}
\href{http://jpr.sagepub.com/content/49/2/363}{Tollefsen, Andreas Forø;
Håvard Strand \& Halvard Buhaug (2012) PRIO-GRID: A unified spatial data
structure. \emph{Journal of Peace Research}, 49(2): 363-374. doi:
10.1177/0022343311431287}
\end{quote}

The article is \emph{open access} and freely available for download at
the journal's web page.

PRIO-GRID consists of data from multiple third-party sources. Hence,
users are requested to cite the original source for each variable used
in their work, in addition to citing the PRIO-GRID article. See variable
descriptions in this codebook for each variable for the correct
citation).

If you want to cite this codebook, please use:

\begin{quote}
Tollefsen, Andreas Forø, Karim Bahgat, Jonas Nordkvelle and Halvard
Buhaug (2015). \emph{PRIO-GRID v.2.0 Codebook}. Peace Research Institute
Oslo.
\end{quote}

\paragraph{Funding:}\label{funding}

PRIO-GRID was initiated by Andreas Forø Tollefsen and Halvard Buhaug in
2008. The project was an integral part of the Advanced Conflict Data
Catalogue (ACDC) project (2011-2013), led by Håvard Strand, and funded
by the Research Council of Norway. The upgrade to PRIO-GRID version 2.0
has been funded by the Research Council of Norway and the European
Research Council through separate research project grants (\#240315-F10
and \#648291, respectively), led by Halvard Buhaug (PRIO).

\paragraph{Acknowledgements:}\label{acknowledgements}

We thank numerous colleagues at PRIO, Uppsala University, ENCoRe Cost
Network, ETH Zürich, University of Colorado Boulder, CIESIN Columbia
University, participants at FOSS4G 2011 conference in Denver, and users
around the world for crucial feedback during the development of the
PRIO-GRID project. Gerdis Wischnath, Johan Dittrich Hallberg, and Nils
Weidmann provided important input on the initial version.

We also appreciate the cooperation with providers of the source data
that are integrated into the PRIO-GRID framework. Please do cite the
sources as described in relation to each variable.

\paragraph{Questions and Support:}\label{questions-and-support}

The PRIO-GRID website can be found at:
\href{http://grid.prio.org}{grid.prio.org}. The website provides an
interactive visualization of the PRIO-GRID variables and allows users to
download variables of interest as spreadsheets or shapefiles through the
data portal.

Questions and comments should be addressed to Andreas Forø Tollefsen:
\href{mailto:andreas@prio.no}{andreas@prio.org}.

\mainmatter

\chapter{Introduction}\label{introduction}

This document describes the development and content of the PRIO-GRID
dataset, a standardized spatial grid structure with global coverage at a
resolution of 0.5 x 0.5 decimal degrees. See Tollefsen, Strand \& Buhaug
(2012) for additional information on the background, motivation, and
application of PRIO-GRID.

PRIO-GRID consists of four components. The first is the tabular dataset,
containing spatially disaggregated data at the grid cell level. While
these tables do not contain the geometries per se, they can be
represented and visualized using the PRIO-GRID geographic information
systems (GIS) shapefiles, which contain the polygon grid and the
corresponding cell centroids. Two .csv files are available for download;
one .csv table for static variables and one .csv table for time-varying
variables where the grid has one realization per calendar year. The
content of the .csv files depends on the variables selected for download
through the data portal.

The second component includes open-source replication scripts that were
used to generate the PRIO-GRID dataset, publically available through the
GitHub-repository at
\href{http://www.github.com/prio-data/priogrid}{github.com/prio-data/priogrid}.
These files facilitate replication, modification, and extension of the
original files, including joining of additional geo-referenced data,
should the user wish to do so.

The third component is the documentation, consisting of the journal
article presenting PRIO-GRID (Tollefsen, Strand \& Buhaug 2012), this
codebook, and the instructions for how to use the replication files at
\href{http://www.github.com/prio-data/priogrid}{GitHub}.

The fourth component is a shapefile of the grid-cells. The shapefile is
a 0.5 x 0.5 decimal degree grid system of the world. It contains both
land and sea grids. When merged with the data, you can use this
shapefile to plot data on a map.

PRIO-GRID is a versioned dataset, meaning that changes to the data are
released with new version numbers. Higher version numbers indicate more
recent data. All files, scripts, and documentation should reflect these
version changes.

\section{Changes in version 2.0}\label{changes-in-version-2.0}

PRIO-GRID version 2.0 introduces several updates, changes, and new
additions since version 1.01. PRIO-GRID 2.0 extends the temporal
coverage until 2014, providing one annual grid representation of the
globe for each year, 1946-2014. While previous PRIO-GRID versions only
contained grid cells occupied by independent states as defined by the
Gleditsch \& Ward system membership list, version 2.0 contains all
terrestrial cells, in total 64,818 per grid, for all years regardless of
the political status of the territory. For grid cells covering
non-independent territory, the country code (gwno) will be missing. This
is done to facilitate inclusion of time-series data that are not
associated with a country, notably climate statistics.

\paragraph{List of changes}\label{list-of-changes}

\begin{itemize}
\itemsep1pt\parskip0pt\parsep0pt
\item
  The data structure has been refactored for easier use. Version 2.0
  consists of one static table, one yearly table, and one shapefile with
  the corresponding geometries. The content of the tables will be
  determined by the user when generating the data through the data
  portal.
\item
  Duplicates of cell-year observations (previously done to permit info
  on overlapping ethnic groups) have been removed. Instead, link tables,
  such as GeoEPR2PRIO-GRID, are provided as an extension.
\item
  A new interactive data portal has been created where PRIO-GRID data
  can be visualized, queried, and downloaded.
\item
  The development process has been made fully automated and replicable
  in a downloadable package, along with instructions for how to use.
  This includes converting from a mixed use of Python and SQL scripts to
  nearly pure SQL scripts, for less dependencies and easier replication.
  Any replication will require a PostGIS compatible database. The
  scripts used to create PRIO-GRID have been made available as
  open-source files through the GitHub data repository.
\item
  In addition to general updating of time-varying variables to cover
  more recent years and minor adjustments in some variable
  operationalizations (see data description below for details), several
  new indicators have been added, and some have been removed. More
  specifically:

  \begin{itemize}
  \itemsep1pt\parskip0pt\parsep0pt
  \item
    The Conflict Sites and onset data featured in v.1.01 are not
    included in v.2.0 since the underlying Conflict Site dataset has not
    been updated after 2008. Please refer back to v.1.01 for these data
    (merge using gid-year). Note that the UCDP Georeferenced Event Data,
    UCDP conflict polygons, and the ACLED datasets provide links to
    PRIO-GRID cell IDs for each event and can thus be easily imported
    into the grid.
  \item
    All distance measurements are now spherical distances, rather than
    geometric distances.
  \item
    Rather than listing all GeoEPR groups within a cell, v.2.0 includes
    a count of \nameref{excluded} groups within a gid-year. A new link
    table, GeoEPR2PRIO-GRID, is available on the PRIO-GRID web as an
    extension for users who want to import additional information from
    the EPR family datasets.
  \item
    Globcover landuse coverages are now separated into variables
    (\nameref{urban-gc}, \nameref{agri-gc}, \nameref{forest-gc},
    \nameref{shrub-gc}, \nameref{herb-gc}, \nameref{aquaveg-gc},
    \nameref{barren-gc}, \nameref{water-gc}), rather than using the
    combination of \textbf{lclass} and \textbf{lclasspct} in PRIO-GRID
    v.1.01. Users wishing to aggregate their own combinations of landuse
    types should consult the old v.1.01.
  \item
    Data on irrigation (\nameref{irrig-}) is now based on a new data
    source that also captures changes over time.
  \item
    Precipitation data (\nameref{prec-gpcc} and \nameref{prec-gpcp}) are
    now derived from two alternative sources, GPCC and GPCP, rather than
    the University of Delaware (NOAA 2011) data provided through
    PRIO-GRID v.1.01.
  \item
    Temperature data (\nameref{temp}) are now derived from GHCN/CAMS (Fan
    et.al. 2008), rather than the University of Delaware (NOAA 2011)
    data provided through PRIO-GRID v.1.01.
  \item
    A number of new drought measures have been added.
  \item
    Data on the location of diamond, petroleum, gems, gold, and drugs
    deposits have been added.
  \item
    Data on child malnutrition has been added (CIESIN CMR).
  \item
    Satellite Nightlight emission data have been added (DMSP-OLS).
  \item
    Crop and landuse data from MIRCA2000 and ISAM-HYDE have been added.
  \item
    Population data from HYDE have been added to supplement the GPW
    data.
  \end{itemize}
\end{itemize}

\section{The development of
PRIO-GRID}\label{the-development-of-prio-grid}

PRIO-GRID is generated in a relational database management system
(RDBMS); PostgreSQL with the spatial PostGIS extension supplying the
geometric functionality of the Structured Query Language (SQL) database.
PRIO-GRID is released with a 0.5 x 0.5 decimal degree cell resolution.
This corresponds to a cell of roughly 55 x 55 kilometers at the Equator
(3025 square kilometers area). Cell area decreases at higher latitudes
due to the curvature of the earth.

The grid structure is defined by a south-western starting point defined
by x and y coordinates (90S and 180W) and represented using the WGS84
geographic reference system. The cell identifier starts at 1 at the
south-western corner (column 1 and row 1) and increases by 1 for each
column, until reaching 720 (column 720 and row 1). The cell identifier
then starts at the next row and begins at 721 (column 1 row 2). The full
grid at 0.5 x 0.5 degrees resolution contains 259,200 cells (720 x 360).
A majority of these cells cover water and other uninhabited areas
(notably the Arctic and Antarctica) and are of little relevance in most
applications. To limit file size, the released PRIO-GRID only includes
terrestrial grid cells (64,818) although the full grid is maintained and
is available on request. The current version of PRIO-GRID consists of
one grid per calendar year for the period 1946--2014.

The remaining sections of the codebook contain a brief presentation of
all variables in the PRIO-GRID files and how they were imported and
modified to fit into the PRIO-GRID data structure.

\section{The grid reference file}\label{the-grid-reference-file}

The grid reference file contains information about the PRIO-GRID spatial
data structure. This file is provided in the ESRI shapefile format,
where each cell is represented by a rectangular vector geometry in
addition to a shapefile containing the centroid point. Variables from
the static and temporal files can be visualized and analyzed by merging
data to the shapefile via the grid identifier (see below).

\section{Adding additional data using the provided
shapefile}\label{adding-additional-data-using-the-provided-shapefile}

In addition to the data available in the tables explained above, we
provide a shapefile with the cell geometry that make it possible for
users to add their own data. This file may be used in a GIS software to
extract, join or overlay with other spatial data. The shapefile may be
joined to the various attribute tables using the gid variable.

\section{Using the replication
scripts}\label{using-the-replication-scripts}

PRIO-GRID aims to be transparent and is fully replicable with a set of
automated script. The necessary files and instructions can be found at
the GitHub repository at
\href{http://www.github.com/prio-data/priogrid}{github.com/prio-data/priogrid}.

\chapter{Overview of Included Data}\label{overview-of-included-data}

This section presents the data available through the PRIO-GRID, and
descriptions of the variable names. Below each source of data is a
reference to the appropriate citation for each data source. Please cite
the original source in addition to the JPR article and this codebook
whenever using PRIO-GRID.

\begin{itemize}
\itemsep1pt\parskip0pt\parsep0pt
\item
  \nameref{grid-cell-identifiers}
\item
  \nameref{the-static-table}

  \begin{itemize}
  \itemsep1pt\parskip0pt\parsep0pt
  \item
    \nameref{accessibility-variables}
  \item
    \nameref{socioeconomic-variables}
  \item
    \nameref{resource-variables}
  \item
    \nameref{landuse-variables}
  \item
    \nameref{climate-variables}
  \end{itemize}
\item
  \nameref{the-temporal-table}

  \begin{itemize}
  \itemsep1pt\parskip0pt\parsep0pt
  \item
    \nameref{country-variables}
  \item
    \nameref{socioeconomic-variables-1}
  \item
    \nameref{resource-variables-1}
  \item
    \nameref{climate-variables-1}
  \item
    \nameref{landuse-variables-1}
  \end{itemize}
\end{itemize}

\section{Grid Cell Identifiers}\label{grid-cell-identifiers}

\paragraph{gid}\label{gid}

is the grid cell identifier, a unique id code for each cell in the grid.
Since we only include the terrestrial cells from the full grid, the gid
starts at 49182 and ends at 249344. See \nameref{the-development-of-prio-grid} for explanation of the grid structure.

\paragraph{col}\label{col}

denotes column number for the grid cell. Column 1 is the westernmost
column in the grid, between 180 and 179.5 decimal degrees W. With one
column per half degree, there are 720 columns in PRIO-GRID.

\paragraph{row}\label{row}

denotes the row number for the grid cell. Row 1 is the southernmost row
(between 90 and 89.5 degrees S) and row 360 is the northernmost row in
the full grid in the underlying data.

\paragraph{xcoord}\label{xcoord}

denotes the longitude coordinate (decimal degrees) for the centroid of
the grid cell. Negative coordinates are located west of the Prime
Meridian (Greenwich) at 0 degrees longitude.

\paragraph{ycoord}\label{ycoord}

denotes the latitude coordinate (decimal degrees) for the centroid of
the grid cell. Negative coordinates are located south of the Equator at
0 degrees latitude.

\section{The Static Table}\label{the-static-table}

The static table contains a grid cell identifier (gid). This means that
gid constitutes the unique identifier in the static file. The PRIO-GRID
static table contains observations of all terrestrial grid cells (based
on cShapes, thus excluding Antarctica, Greenland, and several smaller
island states). In total, the table contains 64,818 observations.

\paragraph{gid}\label{gid-1}

is the grid cell identifier, a unique id code for each cell in the grid.

\paragraph{landarea}\label{landarea}

gives the total area covered by land in the grid cell in square
kilometers as defined by the CShapes dataset. Hence, we exclude
Antarctica, Greenland, and several smaller island states. Areas are
calculated assuming that the earth in an oblate spheroid (WGS 84).

Please cite:

\begin{quote}
Weidmann, Nils B; Doreen Kuse \& Kristian Skrede Gleditsch (2010) The
geography of the international system: The CShapes Dataset.
\emph{International Interactions}, 36(1): 86-106.
\end{quote}

\subsection{Accessibility variables}\label{accessibility-variables}

\paragraph{ttime\_
{[}\href{http://forobs.jrc.ec.europa.eu/products/gam/}{Original
data}{]}}\label{ttime-}

is an estimate of the travel time to the nearest major city, derived
from a global high-resolution raster map of accessibility developed for
the EU. The original indicator is a result of network analysis using a
combination of
\href{http://forobs.jrc.ec.europa.eu/products/gam/sources.php}{several
sources}, most collected between 1990 and 2005. The original pixel value
is the estimated travel time in minutes by land transportation from the
pixel to the nearest major city with more than 50 000 inhabitants.

\begin{itemize}
\itemsep1pt\parskip0pt\parsep0pt
\item
  \textbf{ttime\_mean} gives the average travel time within each cell.
\item
  \textbf{ttime\_sd} gives the standard deviation of original pixel
  values within each cell.
\item
  \textbf{ttime\_min} gives the minimum original pixel value within each
  cell.
\item
  \textbf{ttime\_max} gives the maximum original pixel value within each
  cell.
\end{itemize}

Please cite:

\begin{quote}
Uchida, Hirotsugu and Nelson, Andrew (2009). Agglomeration Index:
Towards a New Measure of Urban Concentration. \emph{Background paper for
the World Bank's World Development Report 2009}.
\end{quote}

\paragraph{mountain\_mean
{[}\href{http://www.unep-wcmc.org/resources-and-data/mountains-and-forests-in-mountains}{Original
data}{]}}\label{mountain-mean}

measures the proportion of mountainous terrain within the cell based on
elevation, slope and local elevation range, taken from a high-resolution
mountain raster developed for UNEP's Mountain Watch Report. The original
pixel values are binary, capturing whether the pixel is a mountain pixel
or not based on the seven different categories of mountainous terrain in
the report.

Please cite:

\begin{quote}
Blyth, Simon, Brian Groombridge, Igor Lysenko, Lera Miles, and Adrian
Newton (2002). \emph{Mountain Watch: environmental change \& sustainable
development in mountains.} UNEP-WCMC Biodiversity Series 12. ISBN:
1-899628-20-7
\end{quote}

\subsection{Socioeconomic variables}\label{socioeconomic-variables}

\paragraph{imr\_
{[}\href{http://sedac.ciesin.columbia.edu/data/set/povmap-global-subnational-infant-mortality-rates}{Original
data}{]}}\label{imr-}

measures infant mortality rate, based on raster data from the SEDAC
Global Poverty Mapping project. The original pixel value is the number
of children per 10,000 live births that die before reaching their first
birthday. This indicator is a snapshot for the year 2000 only.

\begin{itemize}
\itemsep1pt\parskip0pt\parsep0pt
\item
  \textbf{imr\_mean} gives the average infant mortality rate within the
  grid cell.
\item
  \textbf{imr\_sd} gives the standard deviation of original pixel values
  within each cell.
\item
  \textbf{imr\_min} gives the minimum of original pixel values within
  each cell.
\item
  \textbf{imr\_max} gives the maximum of original pixel values within
  each cell.
\end{itemize}

Please cite:

\begin{quote}
Storeygard, Adam; Deborah Balk, Marc Levy \& Glenn Deane (2008) The
global distribution of infant mortality: A subnational spatial view.
\emph{Population, Space and Place}, 14(3):209-229.
\end{quote}

\begin{quote}
Center for International Earth Science Information Network - CIESIN -
Columbia University. 2005. \emph{Poverty Mapping Project: Global
Subnational Infant Mortality Rates}. Palisades, NY: NASA Socioeconomic
Data and Applications Center (SEDAC). doi:10.7927/H4PZ56R2. Accessed
19.05.2006.
\end{quote}

\paragraph{cmr\_
{[}\href{http://sedac.ciesin.columbia.edu/data/set/povmap-global-subnational-prevalence-child-malnutrition}{Original
data}{]}}\label{cmr-}

measures the prevalence of child malnutrition, based on raster data from
the SEDAC Global Poverty Mapping project. The original pixel value is
the percent of children under the age of 5 that are malnutritioned. This
indicator is a snapshot for the year 2000 only.

\begin{itemize}
\itemsep1pt\parskip0pt\parsep0pt
\item
  \textbf{cmr\_mean} gives the average prevalence of child malnutrition
  within the grid cell.
\item
  \textbf{cmr\_sd} gives the standard deviation of original pixel values
  within each cell.
\item
  \textbf{cmr\_min} gives the minimum of original pixel values within
  each cell.
\item
  \textbf{cmr\_max} gives the maximum of original pixel values within
  each cell.
\end{itemize}

Please cite:

\begin{quote}
Center for International Earth Science Information Network - CIESIN -
Columbia University. 2005. \emph{Poverty Mapping Project: Global
Subnational Prevalence of Child Malnutrition}. Palisades, NY: NASA
Socioeconomic Data and Applications Center (SEDAC).
doi:10.7927/H4K64G12. Accessed 13.08.2015.
\end{quote}

\subsection{Resource variables}\label{resource-variables}

\paragraph{petroleum\_s
{[}\href{https://www.prio.org/Data/Geographical-and-Resource-Datasets/Petroleum-Dataset/Petroleum-Dataset-v-12/}{Original
data}{]}}\label{petroleum-s}

is a dummy variable for whether onshore petroleum deposits have been
found within the given grid cell, based on the Petroleum Dataset v.1.2.
This variable only codes those petroleum deposits that do not have a
known discovery or start of production year. For a complete picture,
these data must therefore be combined with the \nameref{petroleum-y}
data.

Please cite:

\begin{quote}
Lujala, Päivi, Jan Ketil Rød \& Nadia Thieme, 2007. Fighting over Oil:
Introducing A New Dataset. \emph{Conflict Management and Peace Science},
24(3), 239-256.
\end{quote}

\paragraph{diamsec\_s
{[}\href{https://www.prio.org/Data/Geographical-and-Resource-Datasets/Diamond-Resources/}{Original
data}{]}}\label{diamsec-s}

is a dummy variable for whether secondary (alluvial) diamond deposits
have been found within the given grid cell, based on the Diamond
Resources dataset v1a. This variable only codes those deposits that do
not have a known discovery or start of production year. For a complete
picture, these data must therefore be combined with the
\nameref{diamsec-y} data.

Please cite:

\begin{quote}
Gilmore, Elisabeth, Nils Petter Gleditsch, Päivi Lujala \& Jan Ketil
Rød, 2005. Conflict Diamonds: A New Dataset, \emph{Conflict Management
and Peace Science} 22(3): 257--292
\end{quote}

\begin{quote}
Lujala, Päivi, Nils Petter Gleditsch \& Elisabeth Gilmore, 2005. A
Diamond Curse? Civil War and a Lootable Resource. \emph{Journal of
Conflict Resolution}, 49(4): 538--562.
\end{quote}

\paragraph{diamprim\_s
{[}\href{https://www.prio.org/Data/Geographical-and-Resource-Datasets/Diamond-Resources/}{Original
data}{]}}\label{diamprim-s}

is a dummy variable for whether primary (kimberlite) diamond deposits
have been found within the given grid cell, based on the Diamond
Resources dataset v1a. This variable only codes those deposits that do
not have a known discovery or start of production year. For a complete
picture, these data must therefore be combined with the
\nameref{diamprim-y} data.

Please cite the same source as \nameref{diamsec-s}.

\paragraph{goldplacer\_s
{[}\href{http://www.researchgate.net/profile/Sara_Balestri}{Original
data}{]}}\label{goldplacer-s}

is a dummy variable for whether placer gold deposits have been found
within the given grid cell, based on the GOLDATA\_L subset of the
GOLDDATA v1.2. This variable only codes those deposits that do not have
a known discovery or start of production year. For a complete picture,
these data must therefore be combined with the \nameref{goldplacer-y}
data.

Please cite:

\begin{quote}
Balestri, Sara, 2015. GOLDATA: The Gold deposits dataset codebook,
Version 1.2. \emph{UCSC-Cognitive Science and Communication Research Centre}
WP 02/15, Milan. doi:10.13140/RG.2.1.1730.8648
\end{quote}

\begin{quote}
Balestri, Sara, 2012. Gold and civil conflict intensity: evidence from a
spatially disaggregated analysis, Peace Economics. \emph{Peace Science
and Public Policy}, 18(3): 1-17. doi:10.1515/peps-2012-0012.
\end{quote}

\paragraph{goldsurface\_s
{[}\href{http://www.researchgate.net/profile/Sara_Balestri}{Original
data}{]}}\label{goldsurface-s}

is a dummy variable for whether surface gold deposits have been found
within the given grid cell, based on the GOLDATA\_S subset of the
GOLDDATA v1.2. Surface gold deposits are defined as deposits that are
located near the surface but ``do not hold enough information to be
properly defined as lootable {[}placer gold{]}''. This variable only
codes those deposits that do not have a known discovery or start of
production year. For a complete picture, these data must therefore be
combined with the \nameref{goldsurface-y} data.

Please cite the same source as \nameref{goldplacer-s}.

\paragraph{goldvein\_s
{[}\href{http://www.researchgate.net/profile/Sara_Balestri}{Original
data}{]}}\label{goldvein-s}

is a dummy variable for whether vein gold deposits have been found
within the given grid cell, based on the GOLDATA\_NL subset of the
GOLDDATA v1.2. This variable only codes those deposits that do not have
a known discovery or start of production year. For a complete picture,
these data must therefore be combined with the \nameref{goldvein-y}
data.

Please cite the same source as \nameref{goldplacer-s}.

\paragraph{gem\_s
{[}\href{http://paivilujala.weebly.com/gemdata.html}{Original
data}{]}}\label{gem-s}

is a dummy variable for whether gem deposits have been found within the
given grid cell, based on the GEMDATA dataset. This variable only codes
those deposits that do not have a known discovery or start of production
year. For a complete picture, these data must therefore be combined with
the \nameref{gem-y} data.

Please cite:

\begin{quote}
Lujala, Päivi 2009. Deadly Combat over Natural Resources: Gems,
Petroleum, Drugs, and the Severity of Armed Civil Conflict.
\emph{Journal of Conflict Resolution}, 53(1): 50-71.
\end{quote}

\subsection{Landuse variables}\label{landuse-variables}

\paragraph{urban\_gc
{[}\href{http://due.esrin.esa.int/page_globcover.php}{Original
data}{]}}\label{urban-gc}

measures the coverage of urban areas in each cell, based on the
Globcover 2009 dataset v.2.3. To compute \nameref{urban-gc} we follow
the FAO land cover classification system used by Globcover and aggregate
to the category ``Artificial areas'' (Landuse class 190). The value
indicates the percentage area of the cell covered by urban area. This
indicator is a snapshot for the year 2009 only.

Please cite:

\begin{quote}
Bontemps, Sophie; Pierre Defourny \& Eric Van Bogaert (2009) Globcover
2009. Products Description and Validation Report. \emph{European Space
Agency}.
(\url{http://due.esrin.esa.int/files/GLOBCOVER2009_Validation_Report_2.2.pdf}).
\end{quote}

\paragraph{agri\_gc
{[}\href{http://due.esrin.esa.int/page_globcover.php}{Original
data}{]}}\label{agri-gc}

measures the coverage of agricultural areas in each cell, extracted from
the Globcover 2009 dataset v.2.3. To compute \nameref{agri-gc} we follow
the FAO land cover classification system used by Globcover and aggregate
to the category ``Cultivated terrestrial areas and managed lands''
(landuse classes 11, 14, 20, 30). The value indicates the percentage
area of the cell covered by agricultural area. This indicator is a
snapshot for the year 2009 only.

Please cite:

\begin{quote}
Bontemps, Sophie; Pierre Defourny \& Eric Van Bogaert (2009) Globcover
2009. Products Description and Validation Report. \emph{European Space
Agency}.
(\url{http://due.esrin.esa.int/files/GLOBCOVER2009_Validation_Report_2.2.pdf}).
\end{quote}

\paragraph{forest\_gc
{[}\href{http://due.esrin.esa.int/page_globcover.php}{Original
data}{]}}\label{forest-gc}

measures the coverage of forest areas in each cell, extracted from the
Globcover 2009 dataset v.2.3. To compute \nameref{forest-gc} we follow
the FAO land cover classification system used by Globcover and aggregate
to the category ``Woody - trees''(landuse classes 40, 50, 60, 70, 80,
90, 100, 110, 120). The value indicates the percentage area of the cell
covered by forested area. This indicator is a snapshot for the year 2009
only.

Please cite:

\begin{quote}
Bontemps, Sophie; Pierre Defourny \& Eric Van Bogaert (2009) Globcover
2009. Products Description and Validation Report. \emph{European Space
Agency}.
(\url{http://due.esrin.esa.int/files/GLOBCOVER2009_Validation_Report_2.2.pdf}).
\end{quote}

\paragraph{shrub\_gc
{[}\href{http://due.esrin.esa.int/page_globcover.php}{Original
data}{]}}\label{shrub-gc}

measures the coverage of shrubland in each cell, extracted from the
Globcover 2009 dataset v.2.3. To compute \nameref{shrub-gc} we follow
the FAO land cover classification system used by Globcover and aggregate
to the category ``Shrub'' (landuse class 130). The value indicates the
percentage area of the cell covered by shrubland. This indicator is a
snapshot for the year 2009 only.

Please cite:

\begin{quote}
Bontemps, Sophie; Pierre Defourny \& Eric Van Bogaert (2009) Globcover
2009. Products Description and Validation Report. \emph{European Space
Agency}.
(\url{http://due.esrin.esa.int/files/GLOBCOVER2009_Validation_Report_2.2.pdf}).
\end{quote}

\paragraph{herb\_gc
{[}\href{http://due.esrin.esa.int/page_globcover.php}{Original
data}{]}}\label{herb-gc}

measures the coverage of herbaceous vegetation and lichens/mosses in
each cell, extracted from the Globcover 2009 dataset v.2.3. To compute
\nameref{herb-gc} we follow the FAO land cover classification system
used by Globcover and aggregate to the category ``Herbaceous'' (landuse
class 140). The value indicates the percentage area of the cell covered
by herbaceous vegetation and lichens/mosses. This indicator is a snapshot for the year 2009 only.

Please cite:

\begin{quote}
Bontemps, Sophie; Pierre Defourny \& Eric Van Bogaert (2009) Globcover
2009. Products Description and Validation Report. \emph{European Space
Agency}.
(\url{http://due.esrin.esa.int/files/GLOBCOVER2009_Validation_Report_2.2.pdf}).
\end{quote}

\paragraph{aquaveg\_gc
{[}\href{http://due.esrin.esa.int/page_globcover.php}{Original
data}{]}}\label{aquaveg-gc}

measures the coverage of aquatic vegetation in each cell, extracted from
the Globcover 2009 dataset v.2.3. To compute \nameref{aquaveg-gc} we
follow the FAO land cover classification system used by Globcover and
aggregate to the category ``Natural and seminatural aquatic vegetation''
(landuse classes 150, 160, 170, 180). The value indicates the percentage
area of the cell covered by aquatic vegetation. This indicator is a snapshot
for the year 2009 only.

Please cite:

\begin{quote}
Bontemps, Sophie; Pierre Defourny \& Eric Van Bogaert (2009) Globcover
2009. Products Description and Validation Report. \emph{European Space
Agency}.
(\url{http://due.esrin.esa.int/files/GLOBCOVER2009_Validation_Report_2.2.pdf}).
\end{quote}

\paragraph{barren\_gc
{[}\href{http://due.esrin.esa.int/page_globcover.php}{Original
data}{]}}\label{barren-gc}

measures the coverage of barren areas in each cell, extracted from the
Globcover 2009 dataset v.2.3. To compute \nameref{barren-gc} we follow
but deviate slightly from the FAO land cover classification system used
by Globcover by aggregating to the category ``Barren'' and also
including the ``Permanent snow and ice'' class (landuse classes 200,
220). The value indicates the percentage area of the cell covered by
barren area. This indicator is a snapshot for the year 2009 only.

Please cite:

\begin{quote}
Bontemps, Sophie; Pierre Defourny \& Eric Van Bogaert (2009) Globcover
2009. Products Description and Validation Report. \emph{European Space
Agency}.
(\url{http://due.esrin.esa.int/files/GLOBCOVER2009_Validation_Report_2.2.pdf}).
\end{quote}

\paragraph{water\_gc
{[}\href{http://due.esrin.esa.int/page_globcover.php}{Original
data}{]}}\label{water-gc}

measures the coverage of water areas in each cell, extracted from the
Globcover 2009 dataset v.2.3. To compute \nameref{water-gc} we follow
but deviate slightly from the FAO land cover classification system used
by Globcover and aggregate only to the ``Natural/Artificial water
bodies'' class excluding the ``Permanent snow and ice'' class (landuse
class 2010). The value indicates the percentage area of the cell covered
by water area. This indicator is a snapshot for the year 2009 only.

Please cite:

\begin{quote}
Bontemps, Sophie; Pierre Defourny \& Eric Van Bogaert (2009) Globcover
2009. Products Description and Validation Report. \emph{European Space
Agency}.
(\url{http://due.esrin.esa.int/files/GLOBCOVER2009_Validation_Report_2.2.pdf}).
\end{quote}

\paragraph{maincrop
{[}\href{http://www.uni-frankfurt.de/45218031}{Original
data}{]}}\label{maincrop}

indicates the main crop code for the cell, based on the Cropping Periods
List data from the MIRCA2000 dataset v.1.1. The main crop is determined
as the subcrop with the highest harvested area for each cell. Crop codes 26
and below are irrigated crops, while 27 and up are rainfed crops. Data
is only a snapshot for the year 2000.

Please cite:

\begin{quote}
Portmann, Felix T., Stefan Siebert \& Petra Döll (2010): MIRCA2000 --
Global monthly irrigated and rainfed crop areas around the year 2000: A
new high-resolution data set for agricultural and hydrological modeling,
\emph{Global Biogeochemical Cycles}, 24, GB 1011,
doi:10.1029/2008GB003435.
\end{quote}

\paragraph{harvarea
{[}\href{http://www.uni-frankfurt.de/45218031}{Original
data}{]}}\label{harvarea}

is the sum of the harvested area (given in hectares) for the cell's main
crop determined in the \nameref{maincrop} variable, based on the Cropping
Periods List data from the MIRCA2000 dataset v.1.1.

Please cite the same source as \nameref{maincrop}.

\subsection{Climate variables}\label{climate-variables}

\paragraph{rainseas
{[}\href{ftp://ftp.dwd.de/pub/data/gpcc/html/fulldata_v7_doi_download.html}{Original
data}{]}}\label{rainseas}

gives the initial month of the rainy season in the cell (values 1-12),
defined as the first of the three consecutive months during a normal
year with the highest total amount of rainfall, calculated on the basis
of the \nameref{prec-gpcc} variable for the 1946-2013 period.

Please cite:

\begin{quote}
Schneider, Udo, Andreas Becker, Peter Finger, Anja Meyer-Christoffer,
Bruno Rudolf and Markus Ziese (2015). \emph{GPCC Full Data Reanalysis
Version 7.0 at 0.5°: Monthly Land-Surface Precipitation from Rain-Gauges
built on GTS-based and Historic Data}.
doi:10.5676/DWD\_GPCC/FD\_M\_V7\_050
\end{quote}

\paragraph{growstart
{[}\href{http://www.uni-frankfurt.de/45218031}{Original
data}{]}}\label{growstart}

provides the starting month of the growing season for the cell's main
crop determined in the \nameref{maincrop} variable, values 1-12, based on
the Cropping Periods List data from the MIRCA2000 dataset v.1.1.

Please cite:

\begin{quote}
Portmann, Felix T., Stefan Siebert \& Petra Döll (2010): MIRCA2000 --
Global monthly irrigated and rainfed crop areas around the year 2000: A
new high-resolution data set for agricultural and hydrological modeling.
\emph{Global Biogeochemical Cycles}, 24, GB 1011,
doi:10.1029/2008GB003435.
\end{quote}

\paragraph{growend
{[}\href{http://www.uni-frankfurt.de/45218031}{Original
data}{]}}\label{growend}

provides the final month of the growing season for the cell's main crop
determined in the \nameref{maincrop} variable, values 1-12, based on the
Cropping Periods List data from the MIRCA2000 dataset v.1.1.

Please cite the same source as \nameref{growstart}.

\section{The Temporal Table}\label{the-temporal-table}

In addition to the grid identifier, the temporal data table also
includes a year variable. This means that gid + year create a unique
identifier in the time-series data. The PRIO-GRID v.2.0 temporal table
contains yearly observations of all terrestrial grid cells (excluding
Antarctica and Greenland) for all calendar years between 1946 and 2014.
In total, the table contains 64,818 cells x 69 years = 4,472,442
observations (cell years) in total. However, all variables are not
available for all years.

\paragraph{gid}\label{gid-2}

is the grid cell identifier, a unique id code for each cell in the grid.

\paragraph{year}\label{year}

gives the calendar year of observation.

\subsection{Country variables}\label{country-variables}

\paragraph{gwno
{[}\href{http://nils.weidmann.ws/projects/cshapes}{Original
data}{]}}\label{gwno}

denotes the numerical country code for the country to which the cell is
allocated, based on the Gleditsch \& Ward system membership list and
cShapes geometries. Each cell is assigned to one and only one country in
each yearly file. To determine country ownership, PRIO-GRID draws on the
cShapes dataset v.0.4-2, last modified 22 Mar 2015. Grid cells that fall
completely within the territory of an independent state are assigned the
corresponding Gleditsch \& Ward country code (gwno). The country code
reflects the status as of 31 December of each year, which means that in
the case of territorial transfer (e.g., from East Pakistan to Bangladesh
in 1971), a cell is given the country code that applies to the status at
the end of the year, 31 December. Grid cells that cover the territory of
two or more independent states (i.e., the cell intersects with multiple
country polygons) are assigned to the country that covers the largest
share of the cell's area. Note that while all terrestrial cells are
included in the yearly file, country codes are assigned to cells only in
those years that the host country is a member of the Gleditsch \& Ward
international system. Missing values imply non-independent territory.

Please cite:

\begin{quote}
Gleditsch, Kristian Skrede \& Michael D. Ward (1999) Interstate system
membership: A revised list of the independent states since 1816.
\emph{International Interactions}, 25: 393-413.
\end{quote}

\begin{quote}
Weidmann, Nils B., Doreen Kuse \& Kristian Skrede Gleditsch (2010) The
geography of the international system: The CShapes Dataset.
\emph{International Interactions}, 36(1): 86-106.
\end{quote}

\paragraph{gwarea}\label{gwarea}

gives the land area in square kilometers of the grid cell belonging to
the allocated country for that year, based on cShapes. Areas are
calculated assuming that the earth in an oblate spheroid (WGS 84).

Please cite:

\begin{quote}
Weidmann, Nils B., Doreen Kuse \& Kristian Skrede Gleditsch (2010) The
geography of the international system: The CShapes Dataset.
\emph{International Interactions}, 36(1): 86-106.
\end{quote}

\paragraph{bdist1}\label{bdist1}

gives the spherical distance in kilometer from the cell centroid to the
border of the nearest land-contiguous neighboring country, based on
country border data using cShapes v.0.4-2. This implies that cells in
e.g. Northern Denmark are measured to the border to Germany even if the
straight-line distance to Norway (across international waters) is
shorter. Cells belonging to island states with no contiguous neighboring
country (e.g., New Zealand) are coded as missing.

Please cite:

\begin{quote}
Weidmann, Nils B., Doreen Kuse \& Kristian Skrede Gleditsch (2010) The
geography of the international system: The CShapes Dataset.
\emph{International Interactions}, 36(1): 86-106.
\end{quote}

\paragraph{bdist2}\label{bdist2}

gives the spherical distance (in kilometer) from the cell centroid to
the border of the nearest neighboring country, regardless of whether the
nearest country is located across international waters. Hence, for cells
belonging to island states (e.g. New Zealand), bdist2 gives the shortest
distance to the nearest land territory of another state.

Please cite:

\begin{quote}
Weidmann, Nils B., Doreen Kuse \& Kristian Skrede Gleditsch (2010) The
geography of the international system: The CShapes Dataset.
\emph{International Interactions}, 36(1): 86-106.
\end{quote}

\paragraph{bdist3}\label{bdist3}

gives the spherical distance (in kilometer) from the cell centroid to
the territorial outline of the country the cell belongs to. For cells
located along a coast and for cells of island states (e.g. New Zealand),
bdist3 measures the shortest straight-line distance to international
waters. By definition, bdist3 can never have higher values than the two
other border distance indicators and for 44 \% of the cell years all
three border distance estimates are identical.

Please cite:

\begin{quote}
Weidmann, Nils B., Doreen Kuse \& Kristian Skrede Gleditsch (2010) The
geography of the international system: The CShapes Dataset.
\emph{International Interactions}, 36(1): 86-106.
\end{quote}

\paragraph{capdist}\label{capdist}

gives the spherical distance in kilometers from the cell centroid to the
national capital city in the corresponding country, based on coordinate
pairs of capital cities derived from the cShapes dataset v.0.4-2. It
captures changes over time wherever relevant. Figure 3 visualizes these
straight-line distances.

Please cite:

\begin{quote}
Weidmann, Nils B., Doreen Kuse \& Kristian Skrede Gleditsch (2010) The
geography of the international system: The CShapes Dataset.
\emph{International Interactions}, 36(1): 86-106.
\end{quote}

\subsection{Socioeconomic variables}\label{socioeconomic-variables-1}

\paragraph{pop\_gpw\_
{[}\href{http://sedac.ciesin.columbia.edu/data/collection/gpw-v3}{Original
data}{]}}\label{pop-gpw-}

measures population size, taken from the Gridded Population of the World
version 3. Population estimates are available for 1990, 1995, 2000, and
2005. The original pixel value is number of persons.

\begin{itemize}
\itemsep1pt\parskip0pt\parsep0pt
\item
  \textbf{pop\_gpw\_sum} gives the sum of pixel values (number of
  persons) within the grid cell. To obtain population density estimates,
  this variable can be divided by \nameref{landarea} in the static table.
\item
  \textbf{pop\_gpw\_sd} gives the standard deviation of original pixel
  values within each cell.
\item
  \textbf{pop\_gpw\_min} gives the minimum of original pixel values
  within each cell.
\item
  \textbf{pop\_gpw\_max} gives the maximum of original pixel values
  within each cell.
\end{itemize}

Please cite:

\begin{quote}
Center for International Earth Science Information Network (CIESIN) and
Centro Internacional de Agricultura Tropical (CIAT) (2005).
\emph{Gridded Population of the World, Version 3 (GPWv3): Population
Count Grid}. Palisades, NY. doi:10.7927/H4639MPP. Accessed 03.06.2013.
\end{quote}

\paragraph{pop\_hyd\_
{[}\href{http://themasites.pbl.nl/tridion/en/themasites/hyde/download/index-2.html}{Original
data}{]}}\label{pop-hyd-}

measures the population size for each populated cell in the grid, taken
from the History Database of the Global Environment (HYDE) version 3.1.
Population estimates are available for 1950, 1960, 1970, 1980, 1990,
2000, and 2005. The original pixel value is number of persons.

\begin{itemize}
\itemsep1pt\parskip0pt\parsep0pt
\item
  \textbf{pop\_hyd\_sum} gives the sum of pixel values (number of
  persons) within the grid cell. To obtain population density estimates,
  this variable can be divided by\nameref{landarea} in the static table.
\item
  \textbf{pop\_hyd\_sd} gives the standard deviation of original pixel
  values within each cell.
\item
  \textbf{pop\_hyd\_min} gives the minimum of original pixel values
  within each cell.
\item
  \textbf{pop\_hyd\_max} gives the maximum of original pixel values
  within each cell.
\end{itemize}

Please cite:

\begin{quote}
Klein Goldewijk, K. , A. Beusen, M. de Vos and G. van Drecht (2011). The
HYDE 3.1 spatially explicit database of human induced land use change
over the past 12,000 years. \emph{Global Ecology and Biogeography},
20(1): 73-86. doi: 10.1111/j.1466-8238.2010.00587.x.
\end{quote}

\begin{quote}
Klein Goldewijk, K. , A. Beusen, and P. Janssen (2010). Long term
dynamic modeling of global population and built-up area in a spatially
explicit way, HYDE 3 .1. \emph{The Holocene}, 20(4):565-573.
doi:10.1177/0959683609356587
\end{quote}

\paragraph{excluded
{[}\href{http://www.icr.ethz.ch/data/geoepr}{Original
data}{]}}\label{excluded}

counts the number of excluded groups (discriminated or powerless) as
defined in the GeoEPR/EPR data on the status and location of politically
relevant ethnic groups settled in the grid cell for the given year,
derived from the GeoEPR/EPR 2014 update 2 dataset.

Please cite:

\begin{quote}
Vogt, Manuel, Nils-Christian Bormann, Seraina Rüegger, Lars-Erik
Cederman, Philipp Hunziker, and Luc Girardin. 2015. ``Integrating Data
on Ethnicity, Geography, and Conflict: The Ethnic Power Relations
Dataset Family.'' \emph{Journal of Conflict Resolution}, 59(7),
1327-1342. doi:10.1177/0022002715591215
\end{quote}

\paragraph{gcp\_mer {[}\href{http://gecon.yale.edu/}{Original
data}{]}}\label{gcp-mer}

indicates the gross cell product, measured in USD, based on the G-Econ
dataset v4.0, last modified May 2011. The original G-Econ data represent
the total economic activity at a 1x1 degree resolution, so when
assigning this to PRIO-GRID we distribute the total value across the
number of contained PRIO-GRID land cells. In border areas, the G-Econ
1x1 degree cells might overlap with PRIO-GRID cells allocated to a
neighboring country. To minimize bias, PRIO-GRID only extracts G-Econ
data for cells that have the same country code as the G-Econ cell
represents. This variable is only available for five-year intervals
since 1990.

Note: The user should be aware of the following special case. Differing
definitions used by G-Econ and PRIO-GRID over the territorial border
between Libya and Chad across all years resulted in no G-Econ cell being
matched with PRIO-GRID's country definitions, leaving a small strip of
cells in the border region with missing GCP data.

Please cite:

\begin{quote}
Nordhaus, William D. (2006) Geography and macroeconomics: New data and
new findings. \emph{Proceedings of the National Academy of Sciences of
the USA}, 103(10): 3510-3517.
\end{quote}

\paragraph{gcp\_ppp {[}\href{http://gecon.yale.edu/}{Original
data}{]}}\label{gcp-ppp}

indicates the gross cell product, measured in USD using
purchasing-power-parity, based on the G-Econ dataset v4.0, last modified
May 2011. Else similar to \nameref{gcp-mer}, but uses USD at
purchasing-power-parity which corrects for each currency's purchasing
power. This variable is only available for 1990, 1995, 2000, and 2005.

Please cite:

\begin{quote}
Nordhaus, William D. (2006) Geography and macroeconomics: New data and
new findings. \emph{Proceedings of the National Academy of Sciences of
the USA}, 103(10): 3510-3517.
\end{quote}

\paragraph{gcp\_qual {[}\href{http://gecon.yale.edu/}{Original
data}{]}}\label{gcp-qual}

indicates the quality of the GCP values, based on the G-Econ dataset
v4.0, last modified May 2011. Quality is a measure of the quality of the
economic data. Quality = 1 for countries for which the data are
consistent, but it does not capture the quality of the underlying
country statistics. In general, quality \textless{} 1 indicates that
there are major inconsistencies in one of the underlying data inputs
into GCP. See the G-Econ definition table, available at
\url{http://gecon.yale.edu/}.

Please cite:

\begin{quote}
Nordhaus, William D. (2006) Geography and macroeconomics: New data and
new findings. \emph{Proceedings of the National Academy of Sciences of
the USA}, 103(10): 3510-3517.
\end{quote}

\subsection{Resource variables}\label{resource-variables-1}

\paragraph{petroleum\_y
{[}\href{https://www.prio.org/Data/Geographical-and-Resource-Datasets/Petroleum-Dataset/Petroleum-Dataset-v-12/}{Original
data}{]}}\label{petroleum-y}

is a dummy variable for whether onshore petroleum deposits have been
found within the given grid cell for any given year, based on the
Petroleum Dataset v.1.2. This variable only codes those petroleum
deposits that have a known discovery or start of production year. For a
complete picture, these data must therefore be combined with the
\nameref{petroleum-s} data.

Please cite:

\begin{quote}
Lujala, Päivi, Jan Ketil Rød \& Nadia Thieme, 2007. Fighting over Oil:
Introducing A New Dataset. \emph{Conflict Management and Peace Science},
24(3), 239-256.
\end{quote}

\paragraph{diamsec\_y
{[}\href{https://www.prio.org/Data/Geographical-and-Resource-Datasets/Diamond-Resources/}{Original
data}{]}}\label{diamsec-y}

is a dummy variable for whether secondary (alluvial) diamond deposits
have been found within the given grid cell for any given year, based on
the Diamond Resources dataset v1a. This variable only codes those
deposits that have a known discovery or start of production year. For a
complete picture, these data must therefore be combined with the
\nameref{diamsec-s} data.

Please cite:

\begin{quote}
Gilmore, Elisabeth, Nils Petter Gleditsch, Päivi Lujala \& Jan Ketil
Rød, 2005. Conflict Diamonds: A New Dataset. \emph{Conflict Management
and Peace Science}, 22(3): 257--292
\end{quote}

\begin{quote}
Lujala, Päivi, Nils Petter Gleditsch \& Elisabeth Gilmore, 2005. A
Diamond Curse? Civil War and a Lootable Resource. \emph{Journal of
Conflict Resolution}, 49(4): 538--562.
\end{quote}

\paragraph{diamprim\_y
{[}\href{https://www.prio.org/Data/Geographical-and-Resource-Datasets/Diamond-Resources/}{Original
data}{]}}\label{diamprim-y}

is a dummy variable for whether primary (kimberlite) diamond deposits
have been found within the given grid cell for any given year, based on
the Diamond Resources dataset v1a. This variable only codes those
deposits that have a known discovery or start of production year. For a
complete picture, these data must therefore be combined with the
\href{}{diamprim\_s} data.

Please cite:

\begin{quote}
Gilmore, Elisabeth, Nils Petter Gleditsch, Päivi Lujala \& Jan Ketil
Rød, 2005. Conflict Diamonds: A New Dataset, \emph{Conflict Management
and Peace Science} 22(3): 257--292
\end{quote}

\begin{quote}
Lujala, Päivi, Nils Petter Gleditsch \& Elisabeth Gilmore, 2005. A
Diamond Curse? Civil War and a Lootable Resource. \emph{Journal of
Conflict Resolution}, 49(4): 538--562.
\end{quote}

\paragraph{goldplacer\_y
{[}\href{http://www.researchgate.net/profile/Sara_Balestri}{Original
data}{]}}\label{goldplacer-y}

is a dummy variable for whether placer gold deposits have been found
within the given grid cell, based on the GOLDATA\_L subset of the
GOLDATA dataset v1.2. This variable only codes those deposits that have
a known discovery or start of production year. For a complete picture,
these data must therefore be combined with the \nameref{goldplacer-s}
data.

Please cite:

\begin{quote}
Balestri, Sara, 2015. GOLDATA: The Gold deposits dataset codebook,
Version 1.2. \emph{UCSC-Cognitive Science and Communication Research Centre},
WP 02/15, Milan. doi:10.13140/RG.2.1.1730.8648
\end{quote}

\begin{quote}
Balestri, Sara, 2012. Gold and civil conflict intensity: evidence from a
spatially disaggregated analysis. \emph{Peace Economics, Peace Science
and Public Policy}, 18(3): 1-17. doi:10.1515/peps-2012-0012.
\end{quote}

\paragraph{goldsurface\_y
{[}\href{http://www.researchgate.net/profile/Sara_Balestri}{Original
data}{]}}\label{goldsurface-y}

is a dummy variable for whether surface gold deposits defined as
deposits that are located near the surface but ``do not hold enough
information to be properly defined as lootable'' have been found within
the given grid cell, based on the GOLDATA\_S subset of the GOLDATA
dataset v1.2. This variable only codes those deposits that have a known
discovery or start of production year. For a complete picture, these
data must therefore be combined with the \nameref{goldsurface-s} data.

Please cite the same source as \nameref{goldplacer-y}.

\paragraph{goldvein\_y
{[}\href{http://www.researchgate.net/profile/Sara_Balestri}{Original
data}{]}}\label{goldvein-y}

is a dummy variable for whether vein gold deposits have been found
within the given grid cell, based on the GOLDATA\_NL subset of the
GOLDATA dataset v1.2. This variable only codes those deposits that have
a known discovery or start of production year. For a complete picture,
these data must therefore be combined with the \nameref{goldvein-s}
data.

Please cite the same source as \nameref{goldplacer-y}.

\paragraph{gem\_y
{[}\href{http://paivilujala.weebly.com/gemdata.html}{Original
data}{]}}\label{gem-y}

is a dummy variable for whether gem deposits have been found within the
given grid cell, based on the GEMDATA dataset. This variable only codes
those deposits that have a known discovery or start of production year.
For a complete picture, these data must therefore be combined with the
\nameref{gem-s} data.

Please cite:

\begin{quote}
Lujala, Päivi 2009. Deadly Combat over Natural Resources: Gems,
Petroleum, Drugs, and the Severity of Armed Civil Conflict.
\emph{Journal of Conflict Resolution}, 53(1): 50-71.
\end{quote}

\paragraph{drug\_y
{[}\href{http://paivilujala.weebly.com/drugdata.html}{Original
data}{]}}\label{drug-y}

is a dummy variable for whether large-scale drug cultivation (coca bush,
opium poppy, or cannabis) is ongoing within the given grid cell, based
on the DRUGDATA dataset.

Please cite:

\begin{quote}
Buhaug, Halvard \& Päivi Lujala 2005. Accounting for Scale: Measuring
Geography in Quantitative Studies of Civil War. \emph{Political
Geography}, 24: 399-418.
\end{quote}

\subsection{Climate variables}\label{climate-variables-1}

\paragraph{prec\_gpcc
{[}\href{ftp://ftp.dwd.de/pub/data/gpcc/html/fulldata_v7_doi_download.html}{Original
data}{]}}\label{prec-gpcc}

gives the yearly total amount of precipitation (in millimeter) in the
cell, based on monthly meteorological statistics from the Global
Precipitation Climatology Centre. This indicator contains data for the
years 1946-2013 in PRIO-GRID (1901/01 - 2013/12 in the original data).

Please cite:

\begin{quote}
Schneider, Udo, Andreas Becker, Peter Finger, Anja Meyer-Christoffer,
Bruno Rudolf and Markus Ziese (2015): GPCC Full Data Reanalysis Version
7.0 at 0.5°: Monthly Land-Surface Precipitation from Rain-Gauges built
on GTS-based and Historic Data. doi: 10.5676/DWD\_GPCC/FD\_M\_V7\_050
\end{quote}

\paragraph{prec\_gpcp
{[}\href{http://www.esrl.noaa.gov/psd/data/gridded/data.gpcp.html}{Original
data}{]}}\label{prec-gpcp}

gives the yearly total amount of precipitation (in millimeter) in the
cell, based on monthly meteorological statistics from the GPCP v.2.2
Combined Precipitation Data Set. Since the original data only reported
the daily average for each month, we multiplied the daily average by the
number of days in each month in order to obtain approximate monthly
totals, from which yearly totals were estimated. This indicator contains
data for the years 1979-2014.

Please cite:

\begin{quote}
Huffman, G.J., D.T. Bolvin, R.F. Adler, 2012, last updated 2012: GPCP
Version 2.2 SG Combined Precipitation Data Set. WDC-A, NCDC, Asheville,
NC. Dataset accessed 26.06.2015 at
\url{ftp://precip.gsfc.nasa.gov/pub/gpcp-v2.2/psg/}
\end{quote}

Also please note when using:

\begin{quote}
The GPCP combined precipitation data were developed and computed by the
NASA/Goddard Space Flight Center's Laboratory for Atmospheres as a
contribution to the GEWEX Global Precipitation Climatology Project.
\end{quote}

\begin{quote}
The GPCP data was provided by the NOAA/OAR/ESRL PSD, Boulder, Colorado,
USA, from their Web site at \url{http://www.esrl.noaa.gov/psd/}.
\end{quote}

\paragraph{temp
{[}\href{http://www.esrl.noaa.gov/psd/data/gridded/data.ghcncams.html}{Original
data}{]}}\label{temp}

gives the yearly mean temperature (in degrees Celsius) in the cell,
based on monthly meteorological statistics from GHCN/CAMS, developed at
the Climate Prediction Center, NOAA/National Weather Service. This
indicator contains data for the years 1948-2014.

Please cite:

\begin{quote}
Fan, Yun and Huug van den Dool (2008), A global monthly land surface air
temperature analysis for 1948-present, \emph{Journal of Geophysical
Research}, 113, D01103, doi:10.1029/2007JD008470.
\end{quote}

Also please note when using:

\begin{quote}
The GHCN Gridded V2 data was provided by the NOAA/OAR/ESRL PSD, Boulder,
Colorado, USA, from their Web site at
\url{http://www.esrl.noaa.gov/psd/}.
\end{quote}

\paragraph{droughtstart\_spi
{[}\href{http://iridl.ldeo.columbia.edu/maproom/Global/Precipitation/SPI.html}{Original
data}{]}}\label{droughtstart-spi}

gives the severity of drought during the first month of the cell's rainy
season, as defined by the \nameref{rainseas} variable. The severity value
is the SPI1 value during the first month of the rainy season. The
monthly SPI1 index measures deviation from long-term normal rainfall for
that month. The values are standardized where deviation estimates less
than 1 standard deviation indicate near normal rainfall.

We use SPI data from the International Research Institute for Climate
and Society at Colombia University, as defined in Guttman (1999). The
SPI values are calculated based on the
\href{http://www.cpc.ncep.noaa.gov/products/global_precip/html/wpage.cams_opi.html}{CAMS\_OPI}
precipitation dataset. This indicator contains data for the years
1979-2014.

Please cite:

\begin{quote}
Guttman, N. B., 1999: Accepting the Standardized Precipitation Index: A
calculation algorithm. Journal of the American Water Resources
Association, 35(2), 311-322.
\end{quote}

\begin{quote}
McKee, Thomas B., Nolan J. Doesken, and John D. Kliest (1993) The
relationship of drought frequency and duration to time scales. \emph{In
Proceedings of the 8th Conference of Applied Climatology}, 17-22
January, Anaheim, CA. American Meteorological Society, Boston, MA.
179-184.
\end{quote}

\paragraph{droughtend\_spi
{[}\href{http://iridl.ldeo.columbia.edu/maproom/Global/Precipitation/SPI.html}{Original
data}{]}}\label{droughtend-spi}

gives the severity of drought for the entirety of the cell's rainy
season, as defined by the \nameref{rainseas} variable. The severity value
is the SPI3 value for the last month the rainy season. For each month,
the monthly SPI3 index measures deviation from long-term normal rainfall
during the three preceding months. A rainy season is defined as the
three consecutive months in which it on average rained the most during a
year in any cell.

Please cite the same source as \nameref{droughtstart-spi}.

\paragraph{droughtyr\_spi
{[}\href{http://iridl.ldeo.columbia.edu/maproom/Global/Precipitation/SPI.html}{Original
data}{]}}\label{droughtyr-spi}

gives the proportion of months out of 12 months that are part of the
longest streak of consecutive months ending in the given year with SPI1
values below -1.5. For a year where the longest consecutive streak of
months below -1.5 is three, the cell will be given a value of 3/12 =
0.25. When the longest streak starts in the previous year, it is only
counted and included in the year in which the streak ended.
Theoretically, the proportion can become higher than 1.

Please cite the same source as \nameref{droughtstart-spi}.

\paragraph{droughtcrop\_spi
{[}\href{http://iridl.ldeo.columbia.edu/maproom/Global/Precipitation/SPI.html}{Original
data}{]}}\label{droughtcrop-spi}

gives the proportion of months in the growing season that are part of
the longest streak of consecutive months in that growing season with
SPI1 values below -1.5. The growing season is the growing season for the
cell's main crop, defined in the MIRCA2000 dataset v.1.1. For growing
seasons that cross 1 January, we define the whole season to belong to
the year in which the season ended. Thus, a year with two consecutive
months below -1.5 during the growing season that started in September
the previous year and ended in March in the current year, is given a
value of 2/8 = 0.25. Each year only have defined one growing season.

Please cite the same source as \nameref{droughtstart-spi}.

\paragraph{droughtstart\_speigdm
{[}\href{http://sac.csic.es/spei/map/maps.html}{Original
data}{]}}\label{droughtstart-speigdm}

This variable is operationalized similarly as
\nameref{droughtstart-spi}, only that instead of using the SPI1, it uses
the Standardized Precipitation and Evapotranspiration Index SPEI1 from
the SPEI Global Drought Monitor, downloaded 15 July 2015. SPEI GDM uses
the GPCC `first guess' product and GHCN/CAMS, while using the
Thornthwaite potential evapotranspiration (PET) estimation.

Please cite:

\begin{quote}
Beguería, Santiago, Sergio M. Vicente-Serrano, Fergus Reig, and Borja
Latorre (2014), Standardized Precipitation Evapotranspiration Index
(SPEI) revisited: parameter fitting, evapotranspiration models, tools,
datasets and drought monitoring. \emph{International Journal of
Climatology}, 34(10): 3001--3023. doi: 10.1002/joc.3887
\end{quote}

\paragraph{droughtend\_speigdm
{[}\href{http://sac.csic.es/spei/map/maps.html}{Original
data}{]}}\label{droughtend-speigdm}

This variable is operationalized similarly as \nameref{droughtend-spi},
only that instead of using the SPI-1, it uses the Standardized
Precipitation and Evapotranspiration Index SPEI-3 from the SPEI Global
Drought Monitor.

Please cite the same source as \nameref{droughtstart-speigdm}.

\paragraph{droughtyr\_speigdm
{[}\href{http://sac.csic.es/spei/map/maps.html}{Original
data}{]}}\label{droughtyr-speigdm}

This variable is operationalized similarly as \nameref{droughtyr-spi},
only that instead of using the SPI-1, it uses the Standardized
Precipitation and Evapotranspiration Index SPEI-1 from the SPEI Global
Drought Monitor.

Please cite the same source as \nameref{droughtstart-speigdm}.

\paragraph{droughtcrop\_speigdm
{[}\href{http://sac.csic.es/spei/map/maps.html}{Original
data}{]}}\label{droughtcrop-speigdm}

This variable is operationalized similarly as \nameref{droughtcrop-spi},
only that instead of using the SPI-1, it uses the Standardized
Precipitation and Evapotranspiration Index SPEI-1 from the SPEI Global
Drought Monitor.

Please cite the same source as \nameref{droughtstart-speigdm}.

\paragraph{droughtstart\_speibase
{[}\href{https://digital.csic.es/handle/10261/104742}{Original
data}{]}}\label{droughtstart-speibase}

This variable is operationalized similarly as
\nameref{droughtstart-spi}, only that instead of using the SPI-1, it
uses the Standardized Precipitation and Evapotranspiration Index SPEI-1
from the SPEIbase v.2.3. SPEIbase is based on precipitation and
potential evapotranspiration from the Climatic Research Unit of
University of East Anglia
\href{http://browse.ceda.ac.uk/browse/badc/cru/data}{CRU v.3.22}. The
PET estimation used by CRU is the Penman-Montheith method, considered
better than the Thornthwaite estimation.

Please cite:

\begin{quote}
Beguería, Santiago, Sergio M. Vicente Serrano, and Marta Angulo-Martínez
(2010). A Multiscalar Global Drought Dataset: The SPEIbase: A New
Gridded Product for the Analysis of Drought Variability and Impacts.
\emph{Bulletin of the American Meteorological Society}, 91 (10):
1351--1356. doi:10.1175/2010BAMS2988.1
\end{quote}

\paragraph{droughtend\_speibase
{[}\href{https://digital.csic.es/handle/10261/104742}{Original
data}{]}}\label{droughtend-speibase}

This variable is operationalized similarly as \nameref{droughtend-spi},
only that instead of using the SPI-1, it uses the Standardized
Precipitation and Evapotranspiration Index SPEI-3 from the SPEIbase
v.2.3. SPEIbase is based on precipitation and potential
evapotranspiration from the Climatic Research Unit of University of East
Anglia \href{http://browse.ceda.ac.uk/browse/badc/cru/data}{CRU v.3.22}.
The PET estimation used by CRU is the Penman-Montheith method,
considered better than the Thornthwaite estimation.

Please cite the same source as \nameref{droughtstart-speibase}.

\paragraph{droughtyr\_speibase
{[}\href{https://digital.csic.es/handle/10261/104742}{Original
data}{]}}\label{droughtyr-speibase}

This variable is operationalized similarly as \nameref{droughtyr-spi},
only that instead of using the SPI-1, it uses the Standardized
Precipitation and Evapotranspiration Index SPEI-1 from the SPEIbase
v.2.3. SPEIbase is based on precipitation and potential
evapotranspiration from the Climatic Research Unit of University of East
Anglia \href{http://browse.ceda.ac.uk/browse/badc/cru/data}{CRU v.3.22}.
The PET estimation used by CRU is the Penman-Montheith method,
considered better than the Thornthwaite estimation.

Please cite the same source as \nameref{droughtstart-speibase}.

\paragraph{droughtcrop\_speibase
{[}\href{https://digital.csic.es/handle/10261/104742}{Original
data}{]}}\label{droughtcrop-speibase}

This variable is operationalized similarly as \nameref{droughtcrop-spi},
only that instead of using the SPI-1, it uses the Standardized
Precipitation and Evapotranspiration Index SPEI-1 from the SPEIbase
v.2.3. SPEIbase is based on precipitation and potential
evapotranspiration from the Climatic Research Unit of University of East
Anglia \href{http://browse.ceda.ac.uk/browse/badc/cru/data}{CRU v.3.22}.
The PET estimation used by CRU is the Penman-Montheith method,
considered better than the Thornthwaite estimation.

Please cite the same source as \nameref{droughtstart-speibase}.

\subsection{Landuse variables}\label{landuse-variables-1}

\paragraph{irrig\_
{[}\href{https://mygeohub.org/publications/8}{Original
data}{]}}\label{irrig-}

measures the area equipped for irrigation within each cell (in
hectares). The data is taken from the Historical Irrigation dataset v.1,
which indicates pixelated data on areas equipped for irrigation across
time. Specifically we used the AEI\_EARTHSTAT\_IR dataset, which reports
irrigation based on subnational sources and Earthstat historical landuse
data. In PRIO-GRID, this indicator is only available for the years 1950,
1960, 1970, 1980, 1985, 1990, 1995, 2000, and 2005.

\begin{itemize}
\itemsep1pt\parskip0pt\parsep0pt
\item
  \textbf{irrig\_sum} gives the total area (in hectares) equipped for
  irrigation within the grid cell.
\item
  \textbf{irrig\_sd} gives the standard deviation of original pixel
  values within each cell.
\item
  \textbf{irrig\_min} gives the minimum of original pixel values within
  each cell.
\item
  \textbf{irrig\_max} gives the maximum of original pixel values within
  each cell.
\end{itemize}

Please cite:

\begin{quote}
Stefan Siebert, Matti Kummu, Miina Porkka, Petra Döll, Navin Ramankutty,
Bridget R. Scanlon (2015). Historical Irrigation Dataset (HID).
\emph{MyGeoHUB}. doi:10.13019/M20599
\end{quote}

\paragraph{urban\_ih
{[}\href{https://www.atmos.illinois.edu/~meiyapp2/datasets.htm}{Original
data}{]}}\label{urban-ih}

gives the percentage area of the cell covered by urban area, based on
ISAM-HYDE landuse data. To measure the coverage of urban areas we
include the percentage urban areas in a cell extracted from the
ISAM-HYDE historical landuse dataset. To compute \nameref{urban-ih} we
follow the land cover classification system used by ISAM-HYDE and
aggregate to the category ``Urban'' (landuse class ``Urban''). In
PRIO-GRID, this indicator is available for the years 1950, 1960, 1970,
1980, 1990, 2000, and 2010.

Please cite:

\begin{quote}
Meiyappan, Prasanth and Atul K. Jain (2012). Three distinct global
estimates of historical land-cover change and land-use conversions for
over 200 years. \emph{Frontiers of Earth Science}, 6(2), 122-139.
doi:10.1007/s11707-012-0314-2.
\end{quote}

\paragraph{agri\_ih
{[}\href{https://www.atmos.illinois.edu/~meiyapp2/datasets.htm}{Original
data}{]}}\label{agri-ih}

gives the percentage area of the cell covered by agricultural area,
based on ISAM-HYDE landuse data. To measure the coverage of agricultural
areas we include the percentage agricultural areas in a cell extracted
from the ISAM-HYDE historical landuse dataset. To compute
\nameref{agri-ih} we follow the land cover classification system used by
ISAM-HYDE and aggregate to the category ``Total cropland'' (landuse
classes ``C3crop'', ``C4crop''). In PRIO-GRID, this indicator is
available for the years 1950, 1960, 1970, 1980, 1990, 2000, and 2010.

Please cite:

\begin{quote}
Meiyappan, Prasanth and Atul K. Jain (2012). Three distinct global
estimates of historical land-cover change and land-use conversions for
over 200 years. \emph{Frontiers of Earth Science}, 6(2), 122-139. doi:
10.1007/s11707-012-0314-2.
\end{quote}

\paragraph{pasture\_ih
{[}\href{https://www.atmos.illinois.edu/~meiyapp2/datasets.htm}{Original
data}{]}}\label{pasture-ih}

gives the percentage area of the cell covered by pasture area, based on
ISAM-HYDE landuse data. To measure the coverage of pasture areas we
include the percentage pasture areas in a cell extracted from the
ISAM-HYDE historical landuse dataset. To compute \nameref{pasture-ih} we
follow the land cover classification system used by ISAM-HYDE and
aggregate to the category ``Total pastureland'' (landuse classes
``C3past'', ``C4past''). In PRIO-GRID, this indicator is available for
the years 1950, 1960, 1970, 1980, 1990, 2000, and 2010.

Please cite:

\begin{quote}
Meiyappan, Prasanth and Atul K. Jain (2012). Three distinct global
estimates of historical land-cover change and land-use conversions for
over 200 years. \emph{Frontiers of Earth Science}, 6(2), 122-139. doi:
10.1007/s11707-012-0314-2.
\end{quote}

\paragraph{forest\_ih
{[}\href{https://www.atmos.illinois.edu/~meiyapp2/datasets.htm}{Original
data}{]}}\label{forest-ih}

gives the percentage area of the cell covered by forest area, based on
ISAM-HYDE landuse data. To measure the coverage of forest areas we
include the percentage forest areas in a cell extracted from the
ISAM-HYDE historical landuse dataset. To compute \nameref{forest-ih} we
follow the land cover classification system used by ISAM-HYDE and
aggregate to the category ``Total forest'' (landuse classes ``TrpEBF'',
``TrpDBF'', ``TmpEBF'', ``TmpENF'', ``TmpDBF'', ``BorENF'', ``BorDNF'',
``SecTrpEBF'', ``SecTrpDBF'', ``SecTmpEBF'', ``SecTmpENF'',
``SecTmpDBF'', ``SecBorENF'', ``SecBorDNF''). In PRIO-GRID, this
indicator is available for the years 1950, 1960, 1970, 1980, 1990, 2000,
and 2010.

Please cite:

\begin{quote}
Meiyappan, Prasanth and Atul K. Jain (2012). Three distinct global
estimates of historical land-cover change and land-use conversions for
over 200 years. \emph{Frontiers of Earth Science}, 6(2), 122-139. doi:
10.1007/s11707-012-0314-2.
\end{quote}

\paragraph{grass\_ih
{[}\href{https://www.atmos.illinois.edu/~meiyapp2/datasets.htm}{Original
data}{]}}\label{grass-ih}

gives the percentage area of the cell covered by grasslands, based on
ISAM-HYDE landuse data. To measure the coverage of grasslands we include
the percentage grassland areas in a cell extracted from the ISAM-HYDE
historical landuse dataset. To compute \nameref{grass-ih} we follow the
land cover classification system used by ISAM-HYDE and aggregate to the
category ``Total grassland''(landuse classes ``C3grass'', ``C4grass'').
In PRIO-GRID, this indicator is available for the years 1950, 1960,
1970, 1980, 1990, 2000, and 2010.

Please cite:

\begin{quote}
Meiyappan, Prasanth and Atul K. Jain (2012). Three distinct global
estimates of historical land-cover change and land-use conversions for
over 200 years. \emph{Frontiers of Earth Science}, 6(2), 122-139. doi:
10.1007/s11707-012-0314-2.
\end{quote}

\paragraph{shrub\_ih
{[}\href{https://www.atmos.illinois.edu/~meiyapp2/datasets.htm}{Original
data}{]}}\label{shrub-ih}

gives the percentage area of the cell covered by shrublands, based on
ISAM-HYDE landuse data. To measure the coverage of shrublands we include
the percentage shrubland areas in a cell extracted from the ISAM-HYDE
historical landuse dataset. To compute \nameref{shrub-ih} we follow the
land cover classification system used by ISAM-HYDE and aggregate to the
category ``Total shrubland''(landuse classes ``Denseshrub'',
``Openshrub''). In PRIO-GRID, this indicator is available for the years
1950, 1960, 1970, 1980, 1990, 2000, and 2010.

Please cite:

\begin{quote}
Meiyappan, Prasanth and Atul K. Jain (2012). Three distinct global
estimates of historical land-cover change and land-use conversions for
over 200 years. \emph{Frontiers of Earth Science}, 6(2), 122-139. doi:
10.1007/s11707-012-0314-2.
\end{quote}

\paragraph{savanna\_ih
{[}\href{https://www.atmos.illinois.edu/~meiyapp2/datasets.htm}{Original
data}{]}}\label{savanna-ih}

gives the percentage area of the cell covered by grasslands, based on
ISAM-HYDE landuse data. To measure the coverage of savanna we include
the percentage savanna areas in a cell extracted from the ISAM-HYDE
historical landuse dataset. To compute \nameref{savanna-ih} we follow
the land cover classification system used by ISAM-HYDE and aggregate to
the category ``Savanna'' (landuse class ``Savanna''). In PRIO-GRID, this
indicator is available for the years 1950, 1960, 1970, 1980, 1990, 2000,
and 2010.

Please cite:

\begin{quote}
Meiyappan, Prasanth and Atul K. Jain (2012). Three distinct global
estimates of historical land-cover change and land-use conversions for
over 200 years. \emph{Frontiers of Earth Science}, 6(2), 122-139. doi:
10.1007/s11707-012-0314-2.
\end{quote}

\paragraph{barren\_ih
{[}\href{https://www.atmos.illinois.edu/~meiyapp2/datasets.htm}{Original
data}{]}}\label{barren-ih}

gives the percentage area of the cell covered by barren area, based on
ISAM-HYDE landuse data. To measure the coverage of barren areas we
include the percentage barren areas in a cell extracted from the
ISAM-HYDE historical landuse dataset. To compute \nameref{barren-ih} we
aggregate using the following lansuse classes: ``Tundra'', ``Desert'',
``PdRI''. In PRIO-GRID, this indicator is available for the years 1950,
1960, 1970, 1980, 1990, 2000, and 2010.

Please cite:

\begin{quote}
Meiyappan, Prasanth and Atul K. Jain (2012). Three distinct global
estimates of historical land-cover change and land-use conversions for
over 200 years. \emph{Frontiers of Earth Science}, 6(2), 122-139. doi:
10.1007/s11707-012-0314-2.
\end{quote}

\paragraph{water\_ih
{[}\href{https://www.atmos.illinois.edu/~meiyapp2/datasets.htm}{Original
data}{]}}\label{water-ih}

gives the percentage area of the cell covered by water area, based on
ISAM-HYDE landuse data. To measure the coverage of water areas we
include the percentage water areas in a cell extracted from the
ISAM-HYDE historical landuse dataset. To compute \nameref{water-ih} we
aggregate using the following landuse class: ``Water''. In PRIO-GRID,
this indicator is available for the years 1950, 1960, 1970, 1980, 1990,
2000, and 2010.

Please cite:

\begin{quote}
Meiyappan, Prasanth and Atul K. Jain (2012). Three distinct global
estimates of historical land-cover change and land-use conversions for
over 200 years. \emph{Frontiers of Earth Science}, 6(2), 122-139. doi:
10.1007/s11707-012-0314-2.
\end{quote}

\paragraph{nlights\_
{[}\href{http://ngdc.noaa.gov/eog/dmsp/downloadV4composites.html}{Original
data}{]}}\label{nlights-}

measures average nighttime light emission from the DMSP-OLS Nighttime
Lights Time Series Version 4 (Average Visible, Stable Lights, \& Cloud
Free Coverages). We use the data gathered from the newest satellites
(F10 in 1992-93, F12 in 1994-1996, and so on). These data are not
calibrated for time-series analysis, but are available from 1992-2013.

\begin{itemize}
\itemsep1pt\parskip0pt\parsep0pt
\item
  \textbf{nlights\_mean} gives the mean night time lights within the
  grid cell.
\item
  \textbf{nlights\_sd} gives the standard deviation of original pixel
  values within each cell.
\item
  \textbf{nlights\_min} gives the minimum of original pixel values
  within each cell.
\item
  \textbf{nlights\_max} gives the maximum of original pixel values
  within each cell.
\end{itemize}

Please cite/note:

\begin{quote}
Image and data processing by NOAA's National Geophysical Data Center.
DMSP data collected by US Air Force Weather Agency.
\end{quote}

\paragraph{nlights\_calib\_mean
{[}\href{http://ngdc.noaa.gov/eog/dmsp/downloadV4composites.html}{Original
data}{]}}\label{nlights-calib-mean}

measures average nighttime light emission from the DMSP-OLS Nighttime
Lights Time Series Version 4 (Average Visible, Stable Lights, \& Cloud
Free Coverages), calibrated to account for intersatellite differences
and interannual sensor decay using calibration values from Elvidge
et.al. (2013). Thus, they might be more suitable for time-series
analysis. Values are standardized to be between 0 and 1, where 1 is the
highest observed value in the time-series, and 0 is the lowest. The
times-series are available from 1992-2012.

Please cite/note:

\begin{quote}
Elvidge, Christopher D., Feng-Chi Hsu, Kimberly E. Baugh and Tilottama
Ghosh (2014). ``National Trends in Satellite Observed Lighting:
1992-2012.'' \emph{Global Urban Monitoring and Assessment Through Earth
Observation}. Ed. Qihao Weng. CRC Press.
\end{quote}

\begin{quote}
Image and data processing by NOAA's National Geophysical Data Center.
DMSP data collected by US Air Force Weather Agency.
\end{quote}



\end{document}
